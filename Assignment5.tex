\documentclass{article}
\usepackage[utf8]{inputenc}

%% Sets page size and margins
\usepackage[a4paper,top=3cm,bottom=2cm,left=3cm,right=3cm,marginparwidth=1.75cm]{geometry}

%% Useful packages
\usepackage{amsmath,amsthm,amssymb,amsfonts}
\usepackage{graphicx}
\usepackage{listings}
\usepackage[colorlinks=true, allcolors=blue]{hyperref}
\usepackage{xcolor}
\usepackage[normalem]{ulem}
\useunder{\uline}{\ul}{}
\usepackage{longtable}

\definecolor{codegreen}{rgb}{0,0.6,0}
\definecolor{codegray}{rgb}{0.5,0.5,0.5}
\definecolor{codepurple}{rgb}{0.58,0,0.82}
\definecolor{backcolour}{rgb}{0.95,0.95,0.92}

\lstdefinestyle{mystyle}{
    backgroundcolor=\color{backcolour},   
    commentstyle=\color{codegreen},
    keywordstyle=\color{magenta},
    numberstyle=\tiny\color{codegray},
    stringstyle=\color{codepurple},
    basicstyle=\ttfamily\normalsize,
    breakatwhitespace=false,         
    breaklines=true,                 
    captionpos=b,                    
    keepspaces=true,                 
    numbers=left,                    
    numbersep=5pt,                  
    showspaces=false,                
    showstringspaces=false,
    showtabs=false,                  
    tabsize=2
}

\lstset{style=mystyle}



\documentclass{article}

\usepackage[utf8]{inputenc}


\documentclass{article}
\usepackage[utf8]{inputenc}

\title{Assignment5}
\author{Shubham Shrivastava }
\date{\today}

\begin{document}

\maketitle

\section*{Question 1:}
Let ABC be a right triangle in which $a = 8, c =
6$ and $∠B = 90◦$. $BD$ is the perpendicular from
$B$ on $AC$ (altitude). The circle through $B$, $C$, $D$
(circumcircle of $4BCD$) is drawn. Construct
the tangents from $A$ to this circle.
\section*{Solution:}
Given,\\
$BC = 8cm$, $AB=6cm$ and  $∠B = 90◦$\\
Steps of constructions\\
1. First of all, construct a triangle $ABC$.\\
$\\$
2. Then, project an altitude to hypotenuse $AC$ which meets it at $D$.\\
$\\$
3. Take midpoint of $BC$ as O, taking O as center and $OB$ as radius, make a circle which passes through $A and C$ and intersects triangle at $D$.\\
$\\$
4.Join $A$ and $O$ and bisect it at $E$.\\
$\\$
5. Taking $E$ as center and $EO$ as radius, make another circle which passes through $B, A$ and intersects first circle at $G$.\\
$\\$
6.Now, Join $AG$, this is the required tangent.\\
$\\$
Justification : If we join $OG$, it would make a right angled triangle because any angle in semi circle is right angle. As $OG$ is radius and is perpendicular to AG. AG has to be tangent.\\
\newpage
\begin{figure}[h!]
    \centering
    \includegraphics[height=400, width=400pts]{Q1.JPG}
    \caption{Figure generated by python}
    \label{fig:my_label}
\end{figure}
\newpage


\section*{Question 2:}
Draw a circle with centre C and radius 3.4.
Draw any chord. Construct the perpendicular
bisector of the chord and examine if it passes
through C.
\section*{Solution}
Steps of constructions\\
1. First of all, construct a circle with centre (0,0).\\
$\\$
2. Then, construct a chord $AB$(we can take any two points, so here we are taking A and B).\\
$\\$
3. Bisect the chord and make a line perpendicular to it.\\
$\\$
4.Here we can see that perpendicular to $AB$ passes through centre $O$.\\
$\\$

\begin{figure}[h!]
    \centering
    \includegraphics[height=400, width=400pts]{q2.JPG}
    \caption{Figure generated by python}
    \label{fig:my_label}
\end{figure}




\end{document}
